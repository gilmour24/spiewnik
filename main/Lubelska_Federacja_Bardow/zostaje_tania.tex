\tytul{Zostaje Tania}
{sł. H. Lewandowska, muz. P. Selim}
{Lubelska Federacja Bardów}
\begin{text}
Rzecz miejsce ma w oberży Młodość\\
Kelnerka Tania tanie wino\\
Z niezwyciężoną wręcz swobodą\\
Polewa chłopcom i dziewczynom

Tania podaje tanie wino\\
Chętnie rozmienia ust aksamit\\
Ona tu sługą i władczynią\\
Oni z jej dłoni motylami

\vin [2x]\\
\vin Tu nikt nie mówi o winie\\
\vin Tu nikt nie mówi o karze\\
\vin Tutaj się uczy jedynie\\
\vin Jak nie wyleczyć się z marzeń

Płoną oddechy płyną słowa\\
Z mętnego nurtu wyobrażeń\\
Wariatka Tania kręci w głowach\\
A myśli syczą od oparzeń

\vin [2x]\\
\vin Tu nikt nie mówi o winie\\
\vin Tu nikt nie mówi o karze\\
\vin Tutaj się uczy jedynie\\
\vin Jak nie wyleczyć się z marzeń

Chłopcy dziewczyny stąd odchodzą\\
Przeważnie idą gdzieś parami\\
Zostaje Tania, tania młodość\\
I ci co wciąż są tacy sami

\vin [2x]\\
\vin Tu nikt nie mówi o winie\\
\vin Tu nikt nie mówi o karze\\
\vin Tutaj się uczy jedynie\\
\vin Jak nie wyleczyć się z marzeń
\end{text}
\begin{chord}
e\\
a/E\\
e\\
C H

e\\
a/E\\
e\\
C H e

\hfill\break
G D/Fis e D/Fis\\
a G D/Fis e\\
C D e\\
C D e D/Fis
\end{chord}
