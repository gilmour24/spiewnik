
\tytul{Piosenka w samą porę}{Szymon Zychowicz, Jan Kondrak}{Lubelska Federacja Bardów}
\begin{text}
    Pozwól odejść już\\
    Że nie całkiem zechciej wierzyć\\
    Pozwól odejść już\\
    Najlepszemu z Twych żołnierzy\\
    Miejsce w szyku znam\\
    Żołnierz mieszka w czasie przeszłym\\
    Gdy w swojej roli ma trwać
	
    Tam we mnie obłoki\\
    Obłoki gęstnieją\\
    Tam dzban przepełniony\\
	lekko się chyli\\
    Tam para danieli\\
	przykrywa się knieją\\
    Noc wróży z nocnych motyli

    \vin Na mnie już pora\\
    \vin Nim słowo za ciasne\\
    \vin Nim gest za obszerny nim karta znaczona\\
    \vin Nim zimna koszula obejmie całunem\\
    \vin Te chwile co w nas jak ikona

    Tam we mnie granica\\
    Granica za cicha\\
    Tam grobla mizerna nadmiaru nie zbiera\\
    Tam strażnik zakłada łach przemytnika\\
    Noc wróży z ręki dżokera
	
	\vin Na mnie już pora...

    Pozwól odejść już\\
    Że nie całkiem możesz wierzyć\\
    Pozwól odejść już\\
    Najlepszemu z twych żołnierzy\\
    Miejsce w szyku znam\\
    Moje miejsce w czasie przeszłym\\
    Gdy w swojej roli mam trwać
	
	\vin Na mnie już pora...

\end{text}
\begin{chord}
    e G\\
    D a\\
    e G\\
    D a\\
    e G\\
    D a\\
    e H^7 e
	
    e G\\
    D a\\
    e G\\
    D a\\
    e G\\
    D a\\
    e H^7 e

    e G\\
    D a\\
    e G D a\\
    e G D a\\
    e H^7 e
\end{chord}