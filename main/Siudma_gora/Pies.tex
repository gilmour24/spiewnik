\tytul{Pies}{sł. muz. Z. Siudy}{Siudma góra}
\begin{text}
Nie skorzystam już nigdy z najlepszej z twoich rad\\
Jestem stary jak najstarsza z moich szmat\\
Złota ze mnie nie wypłuczesz nie został nawet gram\\
Gram dla siebie i czasem - dobrze czuję się sam

Mówiłaś: w niedzielę najlepszy ubierz strój\\
Stój i mocno mnie trzymaj – jesteś przecież mój\\
Zabrałaś wszystko został tylko nasz pies\\
Jest łóżko i krzesło tak naprawdę nie ma mnie\\

I tak patrzę w jego oczy wyszczekana morda mać\\
Macie rację że wierność najlepsza jest u psa\\
Rzucić patyk i kopnąć a czasem ryczeć stój\\
Stój ty głupi kundlu jesteś przecież mój

Nie skorzystam już nigdy z najlepszej z twoich rad\\
Jestem stary jak najstarsza z moich szmat\\
Złota ze mnie nie wypłuczesz nie został nawet gram\\
Gram dla siebie i czasem - dobrze czuję się sam
\end{text}
\begin{chord}
    a $\mathrm{F^{7+}}$ F $\mathrm{E^7}$ a\\

\end{chord}
