%%
%% Author: bartek.rydz
%% 10.02.2019
%%
% Preamble
\tytul{Jesień idzie}{sł. Andrzej Waligórski, muz. Olek Grotowski}{Olek Grotowski}
\begin{textn}
    \hfill\break
    Raz staruszek spacerując w lesie\\
    Ujrzał listek przywiędły i blady\\
    I pomyślał - znowu idzie jesień\\
    Jesień idzie, nie ma na to rady

    \vin I podreptał do domu po dróżce\\
    \vin I powiedział stanąwszy przed chatą\\
    \vin Swojej żonie tak samo staruszce\\
    \vin Jesień idzie, nie ma rady na to.

    Zaś staruszka zmartwiła się szczerze\\
    Zamachała rękami obiema\\
    Musisz zacząć chodzić w pulowerze\\
    Jesień idzie, rady na to nie ma

    \vin Może zrobić się chłodno już jutro\\
    \vin Lub pojutrze, a może za tydzień\\
    \vin Trzeba będzie wyjąć z kufra futro\\
    \vin Nie ma rady, jesień, jesień idzie

    A był sierpień pogoda prześliczna\\
    Wszystko w złocie stało i w zieleni\\
    Prócz staruszków nikt chyba nie myślał\\
    O mającej nastąpić jesieni

    \vin Ale cóż, oni żyli najdłużej\\
    \vin Mieli swoje staruszkowe zasady\\
    \vin I wiedzieli, że prędzej czy później\\
    \vin Jesień przyjdzie, nie ma na to rady
\end{textn}
\begin{chordw}
    \textit{Capo III}\\
    e A^7 e A^7\\
    e A^7 h H^7\\
    e A^7 e A^7\\
    C H^7 e A^7

    C D G e\\
    C D G e\\
    C D G e\\
    C H^7 e A^7

    e A^7 e A^7\\
    e A^7 h H^7\\
    e A^7 e A^7\\
    C H^7 e A^7

    C D G e\\
    C D G e\\
    C D G e\\
    C H^7 e A^7

    e A^7 e A^7\\
    e A^7 h H^7\\
    e A^7 e A^7\\
    C H^7 e A^7

    C D G e\\
    C D G e\\
    C D G e\\
    C H^7 e A^7
\end{chordw}