%%
%% Author: EL PROFESOR
%% %26.09.2020
%%
\tytul{Belgijka (Pieśń o Kowalu)}{muz. Lais, tłum. Jacek Folęga}{Było Nas Trzech}
\begin{text}
    Posłuchajcie moi mili\\
    Jak to pewien kowal żył\\
    Co bolesne swe wspomnienia\\
    Chciał obrócić w proch i pył

    \vin Tylko młot, młot, mu dawał siłę\\
    \vin Tylko młot, ciężki młot, kowadła wierny brat (2x)

    Wszak to było nie do wiary\\
    Że z Francuzką związał się\\
    Która jeszcze w noc poślubną\\
    Czystą wydawała się

    I choć była bardzo piękna\\
    To w jej duszy drzemał wąż\\
    Niczym nie mógł jej dogodzić\\
    Zadowolić nie mógł wciąż

    Nie mógł wypić ani kropli\\
    Nie mógł się z kumplami śmiać\\
    Gdy ze swoją był kobietą\\
    Do ust wina nie mógł brać

    Czemu z nią się związał kowal\\
    Lepiej było wdowcem być\\
    Siedzieć w swojej małej kuźni\\
    I w kowadło młotem bić

\end{text}
\begin{chord}
    a\\
    a\\
    a\\
    a

    F G C G a\\
    F G e a

    a\\
    a\\
    a\\
    a

    a\\
    a\\
    a\\
    a

    a\\
    a\\
    a\\
    a

    a\\
    a\\
    a\\
    a
\end{chord}