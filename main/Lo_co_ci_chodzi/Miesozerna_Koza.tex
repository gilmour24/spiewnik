%%
%% Author: EL PROFESOR
%% %26.09.2020
%%
\tytul{Mięsożerna Koza}{Arkadiusz Imiela}{Ło co Ci chodzi}
\begin{text}
    Pasie się na łące koza nietypowa\\
    Tułów jej normalny, za to dziwna głowa\\
    Bródkę ma zwyczajną, nawet zwykłe rogi\\ 
    A za to z jej pyska wystają dwie nogi
	
    Cóż to są za dziwy, cóż to są za czary?\\
    Przynieśli jej mięso jako część ofiary\\
    Połyka łapczywie, krew jej z pyska kapie\\
    A między kęsami złowieszczo sapie...

    \vin Mięsożerne zwierzę trawy w pysk nie bierze (Koza!)\\
    \vin Koza ta to potwór gorszy niż nowotwór! /x2

    Tygrys szablozęby przy niej jest jak kura\\
    Bo jej kły są większe, grubsza także skóra\\
    Zwierz ten demoniczny pożarł raz rekina\\
    Ale sam podpłynął, więc to jego wina

    Pewien spadochroniarz, zupełnie z przypadku\\
    Niestety na jej łące dokonał upadku\\
    Uciec wszak próbował, biegał z wielką gracją\\
    Daleko nie uciekł i stał się kolacją!

    \vin Mięsożerne zwierzę trawy w pysk nie bierze (Koza!)\\
    \vin Koza ta to potwór gorszy niż nowotwór! /x2

    Raz pewien młodzieniec - jurny, chyży, srogi\\
    W kierunku tej wioski skierował swe nogi\\
    Był to szewc Dratewka, z bajki innej znany\\
    A co do tej kozy miał mordercze plany

    Smoka zabił siarką, kozę chciał toporem\\
    Ale nie dał rady sobie z tym potworem\\
    Morał z tego taki, Panowie i Panie\\
    Mięso dla tej kozy to najlepsze danie

    \vin Mięsożerne zwierzę trawy w pysk nie bierze (Koza!)\\
    \vin Koza ta to potwór gorszy niż nowotwór! /x2

\end{text}
\begin{chord}
    E Fis A E\\
    h G D A\\
    h G h G\\
    D A
	
    E Fis A E\\
    h G D A\\
    h G h G\\
    D A

    E G A E\\
    E G A E

    E Fis A E\\
    h G D A\\
    h G h G\\
    D A
	
    E Fis A E\\
    h G D A\\
    h G h G\\
    D A

    E G A E\\
    E G A E

    E Fis A E\\
    h G D A\\
    h G h G\\
    D A
	
    E Fis A E\\
    h G D A\\
    h G h G\\
    D A

    E G A E\\
    E G A E
\end{chord}