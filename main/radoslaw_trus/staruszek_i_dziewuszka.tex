\tytul{Staruszek i dziewuszka (Jesień idzie)}{sł. Radosław Truś, muz. Olek Grotowski}{Radosław Truś}
\begin{text}
	Raz staruszek spacerując w lesie\\
	ujrzał listek, co sczerwieniał prawie\\
	Nic dziwnego, wszak niedługo jesień,\\
	lecz on wcale nie leżał na trawie.
	
	\vin Bo na trawie leżała dziewuszka,\\
	\vin cała naga, piękna była jak łania\\
	\vin A ten listek, co przywabił staruszka,\\
	\vin delikatnie łono jej przysłaniał.
	
	Ach by teraz zająć miejsce listka\\
	i radośnie zabrać się do dzieła…\\
	Na myśl taką przed oczami wszystko\\
	staruszkowi, co mogło, stanęło.
	
	\vin Ogień uczuć i rozkoszy dreszcze,\\
	\vin kawalkada dziewcząt długa i niewąska…\\
	\vin Mógł staruszek długo marzyć jeszcze,\\
	\vin lecz pod butem trzasnęła gałązka.
	
	Więc dziewuszka spłoszyła się nieco,\\
	zamachała rzęsami jak motyl,\\
	Zaś staruszek zesztywniał jak świeca,\\
	przełknął ślinę i zapytał po tym:
	
	\vin Może jest ci teraz chłodno, niebożę?\\
	\vin Słońce zaszło, wiatr krzakami telepie…\\
	\vin Może ja się obok ciebie położę\\
	\vin i przytulę, zaraz będzie lepiej.
	
	Choć był sierpień i pogoda prześliczna,\\
	wszystko wokół trwało w złocie i zieleni,\\
	To dziewuszka zadrżała, a listek\\
	opadł miękko obok niej na ziemię.
	
	\vin I od razu zrobiło się luźniej,\\
	\vin a staruszek przy dziewuszce legł rady,\\
	\vin Wiedział dobrze, że prędzej czy później\\
	\vin każda zechce, nie ma na to rady...  
\end{text}
\begin{chord}
    e A^7 e A^7\\
    e A^7 h H^7\\
    e A^7 e A^7\\
    C H^7 e A^7

    C D G e\\
    C D G e\\
    C D G e\\
    C H^7 e A^7 

    e A^7 e A^7\\
    e A^7 h H^7\\
    e A^7 e A^7\\
    C H^7 e A^7

    C D G e\\
    C D G e\\
    C D G e\\
    C H^7 e A^7 

    e A^7 e A^7\\
    e A^7 h H^7\\
    e A^7 e A^7\\
    C H^7 e A^7

    C D G e\\
    C D G e\\
    C D G e\\
    C H^7 e A^7 

    e A^7 e A^7\\
    e A^7 h H^7\\
    e A^7 e A^7\\
    C H^7 e A^7

    C D G e\\
    C D G e\\
    C D G e\\
    C H^7 e A^7     
\end{chord}
