\tytul{Dziewczna się bała pogrzebów}{sł. muz. Stanisław Staszewski}{Kult}
\begin{text}
    Tak długo szukać i tak dziwnie nagle znaleźć się\\    
    Choć tak co wiosnę jest, to jednak cud\\
    Noc całą biały diabeł na kieliszka tańczy dnie\\
    W dzień czekasz u starego parku wrót\\ 
    Brzeg oceanu marzeń znaczy połamany płot\\
    Jak kipiel morskich pian uliczka bzów\\
    Lecz nagle pęka cisza, a kto wie ten pozna w lot\\
    Pogrzebny dzwon! no cóż - nie przyjdzie znów 

    \vin Bo ona się bała pogrzebów\\
    \vin Co noc pełzły w spokój jej snów\\
    \vin Wóz czarny ze srebrem\\
    \vin I łzy niepotrzebne\\
    \vin Mdłe drżenie chryzantem i głów\\
    \vin Bo ona się bała pogrzebów\\
    \vin jak pająk po twarzy szedł strach\\
    \vin Gdy chude jak szczapy\\
    \vin Szły złe, karę szkapy\\
    \vin Z czarnymi kitami na łbach

    Gdy przyszła brali stary wóz co długo służył już\\
    Lecz ciągnął jeszcze stówę stary grat\\
    Popękał lakier, brzęczą drzwi, rwie sprzęgło - no to cóż?\\
    Pod krzywym dachem piękniej widać świat\\
    Zjeździli mapę już, na zachód, wschód, od dołu wzwyż\\
    l każdy obcy szlak znajomy był\\
    l tylko kiedy czerń chorągwi gdzieś poprzedzał krzyż\\
    W przecznicę pierwszą z brzegu gnał co sił

    \vin Bo ona się bała pogrzebów...

    To tu, spójrz bliżej, Jeszcze w korze ostry został ślad\\
    l płacze las żywicy gorzką łzą\\
    Brzmiał opon śpiew i gadał silnik, w oknach śmiał się wiatr\\
    Gdy na dnie nocy on całował ją\\
    Tu kres podróży znaczy w krwi rubinach brudny koc\\
    Nie skrywam, przedtem jednak wypił ćwierć\\
    Przez chwilę widział szczęście, w światłach uciekało w mrok\\
    No powiedz, może znasz piękniejszą śmierć?

    \vin A kiedyś się bała pogrzebów\\
    \vin I drżało Jej serce gdy szły\\
    \vin Aż wreszcie ten pogrzeb\\
    \vin We dwoje - jak dobrze\\
    \vin W tym srebrze i czerni się śni

\end{text}
\begin{chord}
    c B Gis G\\
    G\\
    c B Gis G\\
    G c\\
    f c f\\
    B c\\
    f c\\
    G\\

    c f\\
    G c\\
    f\\
    c\\
    G\\
    c f\\
    G c\\
    f\\
    c\\
    G c
\end{chord}
