\tytul{Baranek}{sł. muz. Stanisław Staszewski, sł. ref. wg. Dziadów A. Mickiewicza}{Kult}
\begin{text}
    Ech ci ludzie, to brudne świnie\\    
    Co napletli o mojej dziewczynie\\
    Jakieś bzdury o jej nałogach\\
    No to po prostu litość i trwoga\\ 
    Tak to bywa gdy ktoś zazdrości\\
    Kiedy brak mu własnej miłości\\
    Plotki płodzi, mnie nie zaszkodzi żadne obce zło\\
    Na mój sposób widzieć ją 

    \vin Na głowie kwietny ma wianek\\
    \vin W ręku zielony badylek\\
    \vin A przed nią bieży baranek\\
    \vin A nad nią lata motylek

    Krzywdę robią mojej panience\\
    Opluć chcą ją podli zboczeńcy\\
    Utopić chcą ją w morzu zawiści\\
    Paranoicy, podli sadyści\\
    Utaplani w brudnej rozpuście\\
    A na gębach fałszywy uśmiech\\
    Byle zagnać do swego bagna, ale wara wam\\
    Ja ją przecież lepiej znam

    \vin Na głowie kwietny ma wianek...

    Znów widzieli ją z jakimś chłopem\\
    Znów pojechała do St. Tropez\\
    Znów męczyła się, Boże drogi\\
    Znów na jachtach myła podłogi\\
    Tylko czemu ręce ma białe\\
    Chciałem zapytać, zapomniałem\\
    Ciało kłoniąc skinęła dłonią wsparła skroń o skroń\\
    Znów zapadłem w nią jak w toń 

    \vin Na głowie kwietny na wianek...

    Ach, dziewczyna pięknie się stara\\
    Kosi pieniądz, ma jaguara\\
    Trudno pracę z miłością zgodzić\\
    Rzadziej może do mnie przychodzić\\
    Tylko pyta kryjąc rumieniec\\
    Czemu patrzę jak potępieniec\\
    Czemu zgrzytam, kiedy się pyta czy ma ładny biust\\
    Czemu toczę pianę z ust

    \vin Na głowie kwietny ma wianek...

\end{text}
\begin{chord}
    A\\
    d\\
    A\\
    d\\
    D\\
    g\\
    A d\\
    A d\\

    A d\\
    A d\\
    g d\\
    A d
\end{chord}
