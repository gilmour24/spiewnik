\tytul{Beczka piwa}{sł. A. Andrus, muz. O. Grotowski}{Artur Andrus}
\begin{text}
   Pamiętał tylko tyle że\\
   dziewczyna jego miała włosy blond\\
   Pamiętał że przyjechał skądś\\
   lecz nie pamiętał kiedy ani skąd\\
   Urodził się ale nie wie gdzie,\\
   potrafi pisać i modlić się\\
   Lecz czemu po bułgarsku jeden Bóg\\
   naprawdę tylko wie
    
   Pewnego przedpołudnia gdy\\
   dostojnie wszedł do restauracji Dryf\\
   Zamówił beczkę piwa,\\
   piwo, i osiem piw\\
   Usiadł, zmarszczył krzaczastą brew,\\
   w tętnicach mu zaszlochała krew\\
   A duszą targnął płaczliwie\\
   tęskno rzewny śpiew
    
   \vin La, la, la... | x2

   Pamiętał szczyty gór Tien-Szan,\\
   owieczek stado i ogniska swąd\\
   I Marię Konopnicką znał,\\
   lecz ni cholery nie pamiętał skąd\\
   Tatuaż krył jego tors i bark,\\
   miał długie włosy i krótki kark\\
   I dres z nadrukiem klubu MKS\\
   Podhale Nowy Targ

   W plecaku miał swój cały świat:\\
   kamieni kilka, chleb i dwa naboje\\
   I płytę kompaktową\\
   z przebojami Ich Troje\\
   Wspomniał zapach ojczystych łąk,\\
   rodzinną wieś i przyjaciół krąg\\
   I znów nim targnął płaczliwie\\
   tęskny protest song

   \vin La, la, la... | x2

   Pamiętał że miał braci dwóch,\\
   że jeden mieszkał w Soczi i drugi w Lourdes\\
   Ten młody miał na imię Zdzich,\\
   starszemu na cześć dziadka dali Kurt\\
   Ze szkolnych lat zapamiętał, że\\
   z książek miał ulubione dwie\\
   Z pamięci recytował Cichy Don\\
   i statut ZHP
    
   Po trzeciej nocy w barze Dryf\\
   wyglądał jak targnięty sztormem wrak\\
   Utracił czucie w palcach,\\
   powonienie i smak...

   Ale za to odzyskał wiarę w siebie, odnalazł sens życia, i zdobył grono wiernych przyjaciół a wśród nich:
   17 żołnierzy jednostki wojskowej 1530 Regny, 5 pracowników dróg publicznych okręg Rawa Mazowiecka,
   19 studentów wydziału Inżynierii Sanitarnej Politechniki Krakowskiej,
   z którymi każdego wieczora zasiada przy beczce piwa,
   by wspólnie odśpiewać nową wersję pieśni Góralu czy ci nie żal...
 
   \vin La, la, la... | x2
\end{text}
\begin{chord}
   C\\
   C\\
   C\\
   C G\\
   d G\\
   d G\\
   d G\\
   G C

   C\\
   C\\
   C\\
   C^7 F\\
   F G\\
   C A\\
   d G\\
   G C

   C G F G C

   C\\
   C\\
   C\\
   C G\\
   d G\\
   d G\\
   d G\\
   G C

   C\\
   C\\
   C\\
   C^7 F\\
   F G\\
   C A\\
   d G\\
   G C

   C G F G C

   C\\
   C\\
   C\\
   C G\\
   d G\\
   d G\\
   d G\\
   G C

   C\\
   C\\
   C\\
   C^7 F
\end{chord}