%%
%% Author: bartek.rydz
%% 17.02.2019
%%
% Preamble
\tytul{Muszyna}{sł: Adam Ziemianin, muz: Krzysztof Myszkowski}{Stare Dobre Małżeństwo}
\begin{text}
    Dzień już krótszy o krok burmistrza\\
    Głupi Gienek z lipą znowu gada\\
    Nad rzeką stoi miejscowy Heraklit\\
    Ziemia nosi jeszcze sąsiada

    Panna z dzieckiem szuka ojca w lesie\\
    Tu w miasteczku wszyscy się znamy\\
    Sąsiad dostał list aż z Ameryki\\
    A piekarz jest znowu pijany

    Czasem koń Wielki Wóz przeciągnie po niebie\\
    Czasem pies poskarży się do księżyca\\
    Czasem góra pochyli się nad tobą\\
    Czasem wiatr na trawie zagra bluesa

    Przez miasteczko nie da się na skróty\\
    Przez miasteczko trzeba przejść Rynkiem\\
    Tu cię zaraz wezmą na języki\\
    Bo tu każdy jest coś komuś winien

    Dzień krótszy o dwa kroki burmistrza\\
    I jesień z gór schodzi po leszczynach\\
    Powoli zapada w sen zimowy\\
    Razem ze swym bluesem - Muszyna
\end{text}
\begin{chord}
    G\\
    h\\
    $\mathrm{G^{0}}$\\
    B A

    G\\
    h\\
    $\mathrm{G^{0}}$\\
    B A

    G h $\mathrm{G^{0}}$\\
    B A D\\
    G h $\mathrm{G^{0}}$\\
    B A D
\end{chord}
