%%
%% Author: bartek.rydz
%% 17.02.2019
%%
% Preamble
\tytul{Panna Anna}{sł. Bolesław Leśmian, muz. Krzysztof Myszkowski}{Stare Dobre Małżeństwo}
\begin{text}
\footnotesize{
    Kiedy wieczór gaśnie i ustaje dzienny znój -\\
    Panna Anna właśnie najwabniejszy wdziewa strój.\\
    Palce nurza smukłe w czarnoksięskiej skrzyni mrok,\\
    I wyciąga kukłę, co ma w nic utkwiony wzrok.

    To - jej kochan z drewna, zły, bezmyślny, martwy głuch!\\
    Moc zaklęcia śpiewna wprawia go w istnienia ruch.\\
    On nic nie rozumie, lecz za niego działa - czar...\\
    Panna Anna umie kusić wieczność, trwonić żar...

    W dzień od niego stroni, nocą - wielbi sztywny kark,\\
    Nieugiętość dłoni, natarczywość martwych warg.\\
    'Bóg zapomniał w niebie, że samotna ginę w śnie!\\
    Kogóż mam, prócz ciebie? pieść, bo musisz pieścić mnie!'

    Pieści ją bezdusznie, pieści właśnie tak a tak -\\
    A ona posłusznie całym snem omdlewa wznak.\\
    Śmieszny i niezgrabny, swą drewnianą tężąc dłoń,\\
    Szarpie włos jedwabny, miażdży piersi, krwawi skroń.

    Blada, poraniona panna Anna bólom wbrew\\
    Od rozkoszy kona, błogosławiąc mgłę i krew!\\
    Poprzez nocną ciszę idzie cudny, złoty strach...\\
    A śmierć się kołysze cała w rosach, cała w snach.

    Potem nic nie słychać, jakby ktoś na dany znak\\
    Nie chciał już oddychać - byle istnieć tak a tak...\\
    A gdy świt się czyni - panna Anna dwojgiem rąk\\
    Znów zataja w skrzyni drewnianego sprawcę mąk.

    Sztuczne wpina róże w czarny, ciężki, wonny szal -\\
    I po klawiaturze błądząc dłonią - patrzy w dal..\\
    Dźwięki płyną zdradnie, płyną właśnie tak a tak...\\
    Chyba nikt nie zgadnie - z kim spędziła noc i jak?
}
\end{text}
\begin{chord}
\footnotesize{
    H A^{6} H A^{6}\\
    H A^{6} H A^{6}\\
    H A^{6} H A^{6}\\
    H A^{6} H A^{6}

    Fis E H\\
    Fis E H\\
    Fis E H\\
    Fis E H A^{6} H

    H A^{6} H A^{6}\\
    H A^{6} H A^{6}\\
    H A^{6} H A^{6}\\
    H A^{6} H A^{6}

    Fis E H\\
    Fis E H\\
    Fis E H\\
    Fis E H A^{6} H

    H A^{6} H A^{6}\\
    H A^{6} H A^{6}\\
    H A^{6} H A^{6}\\
    H A^{6} H A^{6}

    Fis E H\\
    Fis E H\\
    Fis E H\\
    Fis E H A^{6} H

    Fis E H\\
    Fis E H\\
    Fis E H\\
    Fis E H A^{6} H
}
\end{chord}