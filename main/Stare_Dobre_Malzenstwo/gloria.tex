\tytul{Gloria}{sł. E. Stachura, muz. K. Myszkowski}{Stare Dobre Małżeństwo}
\begin{text}
    Chwała najsampierw komu\\
    Komu gloria\\
    na wysokościach?\\
    Chwała najsampierw tobie\\
    Trawo przychylna każdemu\\
    Kraino na dół od Edenu\\
    Gloria! Gloria!

    Chwała tobie, słońce\\
    Odyńcu ty samotny,\\
    co wstajesz rano z trzęsawisk nocnych\\
    I w góry bieżysz,\\
    w niebo sam się wzbijasz\\
    I chmury czarne białym kłem przebijasz\\
    I to wszystko bezkrwawo - brawo, brawo\\
    I to wszystko złociste i nikogo nie boli\\
    Gloria! Gloria in excelsis soli!

    Z słońcem pochwalonym\\
    teraz pędźmy razem\\
    Na nim, na odyńcu, galopujmy dalej\\
    Chwała tobie wietrze\\
    Wieczny ty młodziku\\
    Sieroto świata, ulubieńcze losu\\
    Od złego ratuj i kąkoli w zbożu\\
    Łagodnie kołysz tych co są na morzu\\
    Gloria! Gloria in excelsis eoli!

    Z wiatrem pochwalonym\\
    teraz pędźmy społem\\
    Na nim, na koniku, galopujmy polem\\
    Chwała wam ptaszki śpiewające\\
    Chwała wam ryby pluskające\\
    Chwała wam zające na łące\\
    Zakochane w biedronce

    Chwała wam: zimy, wiosny,\\
    lata i jesienie\\
    Chwała temu co bez gniewu idzie\\
    Poprzez śniegi, deszcze, blaski oraz cienie\\
    W piersi pod koszulą - całe jego mienie\\
    Gloria! Gloria!
\end{text}
\begin{chord}
    G e\\
    G a\\
    C D G G^7\\
    C D\\
    e\\
    C D C\\
    D G

    G e\\
    G a\\
    C D G G^7\\
    C D\\
    e\\
    C D C\\
    C D C\\
    C D C\\
    D G

    G e\\
    G a\\
    C D G G^7\\
    C D\\
    e\\
    C D C\\
    C D C\\
    C D C\\
    D G

    G e\\
    G a\\
    C D G G^7\\
    C D\\
    e\\
    C D C\\
    D G

    C D\\
    e\\
    C D C\\
    C D C\\
    C D C\\
    D G
\end{chord}