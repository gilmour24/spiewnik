%%
%% Author: EL Profesor
%% %26.09.2020
%%
\tytul{Pytania syna poety}{Andrzej Janeczko}{Trzeci Oddech Kaczuchy}
\begin{text}
    \vin Na, na, na, na, na, na ,na, na, na, na, na,\\
    \vin Na, na, na, na, na, na ,na, na, na, na, na, na

    Ja poeta i łachmyta milczę kiedy syn mnie pyta,\\
    Drogi ojcze ile trzeba wydać na bochenek chleba\\
    I czy można proszę taty, taki chleb kupić na raty?

    Ja poeta i łachmyta milczę , kiedy syn mnie pyta,\\
    Czy to prawda mój ty stary, że na przykład taki Paryż,\\
    Ma na każdym większym płocie rozlepione bezrobocie?

    \vin Na, na, na, na, na, na ,na, na, na, na, na,\\
    \vin Na, na, na, na, na, na ,na, na, na, na, na, na

    Ja poeta i łachmyta milczę kiedy syn mnie pyta,\\
    Czy sportowa, ostra miłość wpływa zgubnie na otyłość,\\
    Czy "love story” diabła warte musi skończyć się bękartem?

    Ja poeta i łachmyta milczę kiedy syn mnie pyta,\\
    Co jest lepsze w życiu tato, skończyć studia, czy łopatą,\\
    Wrzucać węgiel do kopalni lub wyżymać mózgi w pralni?

    Przestań pytać, bo już nocka,\\
    Zamknij oczy, possij smoczka.\\
    Chcesz? To zmienię ci pieluszkę\\
    I pomodlę się nad łóżkiem,\\
    Za obrońców naszych granic,\\
    Za Reytana i kochanie,\\
    Za najwyższą ludzką rację,\\
    Za umiłowaną patrię, patrię, patrię, patrię.\\
    Na, na, na...
\end{text}
\begin{chord}

    C G a e\\
    F C G G^7

    C G a F C G\\
    C G a F C G\\
    a F C G e a

    C G a F C G\\
    C G a F C G\\
    a F C G e a
 
    C G a e\\
    F C G G^7

    C G a F C G\\
    C G a F C G\\
    a F C G e a

    C G a F C G\\
    C G a F C G\\
    a F C G e a

    C G\\
    a F C G\\
    C G\\
    a F C G\\
    a F C G\\
    a F C G\\
    a F C G\\
    e a

\end{chord}