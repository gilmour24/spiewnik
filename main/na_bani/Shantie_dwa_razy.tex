\tytul{Shantie dwa razy}{sł. Tomasz Borkowski, muz.B. Adamczak}{Na Bani}
\begin{text}
    \begin{scriptTwelve}
    I pora się ruszyć, panie Bartek\\
    Choć czas na nas spadł wielkimi kroplami\\
    Góry są jednak wciąż zachodu warte\\
    No, panie Bartek, jak tam będzie z nami?

    Czas w góry jechać, panie Tomek\\
    Po drodze zajdźmy tedy do Baru Na Stawach\\
    A skoro udało się wyrwać raz z domu\\
    Wolności dwa kufle kudłate ja stawiam!

    Na Stawach Bar święty śnięty ma dziś nastrój\\
    Cały czegoś nieswój, chociaż zmartwychwstały\\
    Czapy w nim teraz jakoś mniej zuchwałe\\
    A piwo za słabe, by rozkołysać miasto

    Ech, co za czasy, lata gonią zimy\\
    W Hawiarskiej Kolibie ponoć prohibicja\\
    Wysoki Zamek do nieba poszedł z dymem\\
    Coraz trudniej znaleźć wytęsknioną przystań

    Dobrze, że chór świerków brzmi tym samym głosem\\
    I buki chylą czoła w cerkiewnym obrzędzie\\
    Po latach czas chyba pogodzić się z losem\\
    Że Beskid był!\\
    Że Beskid był!\\
    Że Beskid był, lecz już taki sam nie będzie

    Pan, panie Tomek, ciągle jest szczęściarzem\\
    A pan, panie Bartek, bardem pozostanie\\
    Niech shantie rozwiąże język strun gitarze\\
    I znowu, jak dawniej, zagramy Na Bani!

    Te góry jakieś takie, panie Bartek\\
    I coś tu, panie dzieju, dzieje się źle\\
    Czyżby jesień wszystko zdmuchnęła już całkiem?\\
    Miejmy, panie Bartek, nadzieję że nie!

    Coś, panie Tomek, plecak dzisiaj cięższy\\
    I wyżej niż kiedyś stoi to schronisko\\
    Młodzież w nim siedzi jakaś zbyt dzisiejsza\\
    W naszych czasach lepiej wyglądało wszystko!

    Dziadek z Przegibka odszedł na tułaczkę\\
    Na szczęście w Radocynie twarze wciąż te same\\
    I komin nadal dymi w Surowicznych Polanach\\
    Inaczej przyszłoby ruszyć w ślad za Dziadkiem!

    Ech, co za czasy, lata gonią zimy\\
    W Hawiarskiej Kolibie ponoć prohibicja\\
    Wysoki Zamek do nieba poszedł z dymem\\
    Coraz trudniej znaleźć wytęsknioną przystań

    Dobrze, że chór świerków brzmi tym samym głosem\\
    I buki chylą czoła w cerkiewnym obrzędzie\\
    Po latach czas chyba pogodzić się z losem\\
    Że Beskid był, lecz już taki sam nie będzie

    Pan, panie Tomek, ciągle jest szczęściarzem\\
    A pan, panie Bartek, bardem pozostanie\\
    Niech shantie rozwiąże język strun gitarze\\
    I znowu, jak dawniej, zagramy Na Bani!
\end{scriptTwelve}
\end{text}
\begin{chord}
    \begin{scriptTwelve}
    F $\mathrm{B^9}$ $\mathrm{C^9}$\\
    F $\mathrm{B^9}$ $\mathrm{C^9}$\\
    F $\mathrm{B^9}$ $\mathrm{C^9}$\\
    F $\mathrm{B^9}$ $\mathrm{C^9}$

    \hfill\break
    \hfill\break
    \hfill\break
    \hfill\break
    \hfill\break
    \hfill\break
    \hfill\break
    \hfill\break
    \hfill\break
    $\mathrm{H_7^4}$\\
    E H A\\
    E H A\\
    E H A\\
    E H A

    cis A H\\
    cis A H\\
    cis A H\\
    E H cis\\
    E H cis\\
    E H cis A H

    E H A\\
    E H A\\
    E H A\\
    cis A H C\\
    F $\mathrm{B^9}$ $\mathrm{C^9}$
\end{scriptTwelve}
\end{chord}
