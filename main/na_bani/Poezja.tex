\tytul{Poezja}{sł. W. Broniewski}{Na Bani}
\begin{text}
    \hfill\break
    Ty przychodzisz jak noc majowa,\\
    biała noc, uśpiona w jaśminie,\\
    i jaśminem pachną twoje słowa,\\
    i księżycem sen srebrny płynie,

    Płyniesz cicha przez noce bezsenne\\
    - cichą nocą tak liście szeleszczą-\\
    szepcesz sny, szepcesz słowa tajemne,\\
    w słowach cichych skąpana jak w deszczu
    
    To za mało! Za mało! Za mało!\\
    Twoje słowa tumanią i kłamią!\\
    Piersiom żywych daj oddech zapału,\\
    wiew szeroki i skrzydła do ramion!
    
    Nam te słowa ciche nie starczą.\\
    Marne słowa. I błahe. I zimne.\\
    Ty masz werbel nam zagrać do marszu!\\
    Smagać słowem! Bić pieśnią! Wznieść hymnem!
    
    Jest gdzieś radość ludzka, zwyczajna,\\
    jest gdzieś jasne i piękne życie.\\
    Powszedniego chleba słów daj nam\\
    i stań przy nas, i rozkaż - bić się!
    
    Niepotrzebne nam białe westalki,\\
    noc nie zdławi świętego ognia\\
    bądź jak sztandar rozwiany wśród walki,\\
    bądź jak w wichrze wzniesiona pochodnia!
    
    Jeśli w pięści potrzebna ci harfa,\\
    jeśli harfa ma zakląć pioruny,\\
    rozkaż żyły na struny wyszarpać\\
    i naciągać, i trącać jak struny.

    Odmień, odmień nam słowa na wargach,\\
    naucz śpiewać płomienniej i prościej,\\
    niech nas miłość ogromna potarga.\\
    Więcej bólu i więcej radości!

    Trzeba pieśnią bić aż do śmierci,\\
    trzeba głuszyć w ciemnościach syk węży.\\
    Jest gdzieś życie piękniejsze od nędzy.\\
    I jest miłość. I ona zwycięży.
    
    Wtenczas daj nam, poezjo, najprostsze\\
    ze słów prostych i z cichych - najcichsze,\\
    a umarłych w wieczności rozpostrzyj\\
    jak chorągwie podarte na wichrze.
\end{text}
\hfill
\begin{chord}
    Capo II\\
    a e\\
    F G\\
    a e\\
    F G

    a e\\
    F G\\
    a e\\
    F G
    
    a e\\
    F G\\
    a e\\
    F G

    a e\\
    F G\\
    a e\\
    F G
    
    a e\\
    F G\\
    a e\\
    F G
    
    a e\\
    F G\\
    a e\\
    F G

    a e\\
    F G\\
    a e\\
    F G
    
    a e\\
    F G\\
    a e\\
    F G
    
    a e\\
    F G\\
    a e\\
    F G

    a e\\
    F G\\
    a e\\
    F G
\end{chord}