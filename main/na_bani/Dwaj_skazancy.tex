\tytul{Dwaj Skazancy}{}{Na Bani}
\begin{text}
    Widziałem dwóch skazańców, co na swym uboczu\\
    Wysłuchali wyroku pod bagnetów strażą\\
    I na tłum zgromadzony patrzyli bez oczu,\\
    Jak ślepiec, kiedy zmierzchu wypatruje twarzą.\\
    Jeden z nich, licząc jakieś ubiegłe godziny,\\
    O widzenie się z ojcem poprosił nieśmiało,\\
    A drugi wnet zawołał: 'Ja nie mam rodziny!'\\
    A miał ją, lecz mieć nie chciał... Tak mu się zdawało.\\

    Śnili teraz, że chata, niegdyś ludna, traci\\
    Ich ciała, bezpowrotnie wyszłe z jej alkierza.\\
    Czuli próżnię na miarę wzrostu swych postaci,\\
    Jak klatka, z której nagle wypłoszono zwierza.\\
    Jeden z nich, zapatrzony w strzęp własnego cienia,\\
    Chciwie wody zażądał wargą obolałą,\\
    A drugi wnet zawołał: 'Ja nie mam pragnienia!'\\
    A miał je, lecz mieć nie chciał... Tak mu się zdawało.
\end{text}
