%%
%% Author: bartek.rydz
%% 07.02.2019
%%
% Preamble
\tytul{Tarworthy}{tłum. Jacek Zajączkowski, muz. ludowa}{Krewni i znajomi Królika}
\begin{text}
\footnotesize{
    Żegnaj brzegu Tarworthy adieu Mormon Hill\\
    Biały ptaku Cremontu w morze czas wyjść\\
    Białe skrzydła ma kliper lekko pieści je wiatr\\
    Niespokojny jak mewa chciałby ruszyć w jej ślad

    Żegnajcie kamraci rozstania już czas\\
    Biała Pani Północy białą mgłą wabi nas\\
    Nie ochłodzą mi serca słone lodowe łzy\\
    Im mnie dłużej nie biędzie większy żar spłonie w nim

    Tam u brzegów Grenlandii biała czai się śmierć\\
    Może wrócę bogaty wrócę albo i nie\\
    Z niepokojem już czekam by napełnić swój trzos\\
    W morzu gotów do walki czeka wal na swój los

    Zimny brzeg jest w Grenlandii w lodach czai się zło\\
    Nie ma miejsca i portu o którym wiedziałby ktoś\\
    Och jak pięknie w Tarworthy śpiewa lądowy ptak\\
    Wielorybom o śmierci gwiżdże północny wiatr

    Niedźwiedź król śnieżnych lądów\\
    \vin \vin \vin \vin \vin grozą zdławi nam krzyk\\
    Niepokojem ugości rykiem wedrze się w sny\\
    Gdy ładownie wypełni złoto grenlandzkich mórz\\
    Czas powrotu nadejdzie nic nie wstrzyma nas już
}
\end{text}
\begin{chord}
\footnotesize{
    E cis^7 A E\\
    E cis^7 A H E\\
    E cis^7 dis^7 E\\
    E cis^7 A fis HE

    E cis^7 A E\\
    E cis^7 A H E\\
    E cis^7 dis^7 E\\
    E cis^7 A fis HE

    E cis^7 A E\\
    E cis^7 A H E\\
    E cis^7 dis^7 E\\
    E cis^7 A fis HE

    E cis^7 A E\\
    E cis^7 A H E\\
    E cis^7 dis^7 E\\
    E cis^7 A fis HE

    E cis^7 A E

    E cis^7 A H E\\
    E cis^7 dis^7 E\\
    E cis^7 A fis HE
}
\end{chord}