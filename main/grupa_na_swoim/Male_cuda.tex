%%
%% Author: EL PROFESOR
%% %28.09.2020
%%
\tytul{Małe cuda}{sł. muz. Arkadiusz Piechowski}{Grupa na Swoim}
\begin{text}
    Para para pam...
	
    Dziś na pociąg się spóźniłem, w aucie gumy dwie przebiłem\\
    A biletów na samolot od tygodnia nie ma ponoć\\
    Pajęczyna na rowerze, na niej jakieś groźne zwierzę\\
    Patrzy na mnie srogim okiem, lepiej pójdę na piechotę

    Para para pam...

    I poszedłem leśną drogą, noga człapu człap za nogą\\
    Droga trudna i daleka, po co zwlekać czas ucieka.\\
    Idę wolno i dokładnie, słonko z góry świeci ładnie\\
    I dostrzegłem mały cud, potem drugi, trzeci cud...

    Ślimak stanął mi na drodze na swej jednej śliskiej nodze\\
    Dawaj sera na pierogi, bo Cię wezmę na swe rogi\\
    Ptaki wzięły mnie w obronę: On jest nowy, odpuść sobie!\\
    Kwiaty liśćmi pomachały i z radości pokraśniały

    Mrówka krzyczy ej niecnoty, weźcie wy się do roboty\\
    Pszczoła głową kiwa tylko, bo ma buzię pełną pyłku.\\
    Zapach trawy tańczy wokół, gdzieś nad łąką wisi sokół\\
    Wiatr ciekawski w drzewach lata plotki liściom opowiada

    Wokół mamy małe cuda, rzadko dostrzec się je uda\\
    Czasem zdepcze mały cud jakiś zabiegany but\\
    W maleńkości swej wspaniałe cuda małe można znaleźć\\
    Gdy będziemy wolniej kroczyć kiedy otworzymy oczy

    Gdy wróciłem umęczony czekał już stół zastawiony\\
    Szepcze wróżka mi do uszka: może pójdziesz już do łóżka\\
    Kiedy ściągam swe ubranie to z miłością patrzy na mnie\\
    A gdy przy mnie leży tu to jeszcze jeden mały cud

\end{text}
\begin{chord}
    D A h fis\\
	G D G A
\end{chord}