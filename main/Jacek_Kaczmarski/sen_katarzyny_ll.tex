\tytul{Sen Katarzyny II}{sł. muz. Jacek Kaczmarski}{Jacek Kaczmarski}
\begin{text}
    Na smyczy trzymam filozofów Europy\\
    Podparłam armią marmurowe Piotra stropy\\
    Mam psy, sokoły, konie - kocham łów szalenie\\
    A wokół same zające i jelenie\\
    Pałace stawiam, głowy ścinam\\
    Kiedy mi przyjdzie na to chęć\\
    Mam biografów, portrecistów\\
    I jeszcze jedno pragnę mieć

    \vin - Stój Katarzyno! Koronę Carów\\
    \vin Sen taki jak ten może Ci z głowy zdjąć!

    Kobietą jestem ponad miarę swoich czasów\\
    Nie bawią mnie umizgi bladych lowelasów\\
    Ich miękkich palców dotyk budzi obrzydzenie\\
    Już wolę łowić zające i jelenie\\
    Ze wstydu potem ten i ów\\
    Rzekł o mnie - Niewyżyta Niemra!\\
    I pod batogiem nago biegł\\
    Po śniegu dookoła Kremla!

    \vin - Stój Katarzyno! Koronę Carów\\
    \vin Sen taki jak ten może Ci z głowy zdjąć!

    Kochanka trzeba mi takiego jak imperium\\
    Co by mnie brał tak jak ja daję - całą pełnią\\
    Co by i władcy i poddańca był wcieleniem\\
    I mi zastąpił zające i jelenie\\
    Co by rozumiał tak jak ja\\
    Ten głupi dwór rozdanych ról\\
    I pośród pochylonych głów\\
    Dawał mi rozkosz albo ból!

    \vin - Stój Katarzyno! Koronę Carów\\
    \vin Sen taki jak ten może Ci z głowy zdjąć!\\
    \vin Gdyby się taki kochanek kiedyś znalazł...\\
    \vin - Wiem! Sama wiem! Kazałabym go ściąć!
\end{text}
\begin{chord}
    G D G\\
    G D e\\
    C D e\\
    C D G\\	
    Fis h\\
    Fis G D\\
    C D e\\
    C D G

    e a e a\\
    e a C D G

    G D G\\
    G D e\\
    C D e\\
    C D G\\	
    Fis h\\
    Fis G D\\
    C D e\\
    C D G

    e a e a\\
    e a C D G

    G D G\\
    G D e\\
    C D e\\
    C D G\\	
    Fis h\\
    Fis G D\\
    C D e\\
    C D G

    e a e a\\
    e a C D G\\
    e a e a\\
    e a C D G	
\end{chord}
