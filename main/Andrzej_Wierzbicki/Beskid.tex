%%
%% Author: bartek.rydz
%% 24.05.2018
%%

% Preamble
\tytul{Beskid}{sł. muz. Andrzej Wierzbicki}{SETA}
\begin{text}
    A w Beskidzie rozzłocony buk,\\
    A w Beskidzie rozzłocony buk.\\
    Będę chodził Bukowiną z dłutem w ręku,\\
    By w dziewczęcych twarzach uśmiech rzeźbić\\
    Niech nie płaczą już\\
    Niech się cieszą po kapliczkach moich dróg.

    W Beskidzie, malowany cerkiewny dach,\\
    W Beskidzie, zapach miodu w bukowych pniach.\\
    Tutaj wracam, gdy ruda jesień\\
    Na przełęcze swój tobół niesie,\\
    Słucham bicia dzwonów w przedwieczorny czas.

    W Beskidzie, malowany wiatrami dom,\\
    W Beskidzie, tutaj słowa inaczej brzmią,\\
    Kiedy krzyczą w jesienną ciszę,\\
    Kiedy wiatrem szeleszczą liście,\\
    Kiedy wolność się tuli w ciepło moich rąk,\\
    Gdy wolność się tuli do moich rąk.

    A w Beskidzie zamyślony czas,\\
    A w Beskidzie zamyślony czas.\\
    Będę chodził z moich gór poddaszem,\\
    By zerwanych marzeń struny\\
    Przywiązywać w niepokój i dłonie drzew,\\
    Niech mi grają na rozstajach moich dróg.

    W Beskidzie, malowany cerkiewny dach,\\
    W Beskidzie, zapach miodu w bukowych pniach.\\
    Tutaj wracam, gdy ruda jesień\\
    Na przełęcze swój tobół niesie,\\
    Słucham bicia dzwonów w przedwieczorny czas.
\end{text}
\begin{chord}

\end{chord}