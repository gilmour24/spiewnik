%%
%% Author: bartek.rydz
%% 24.05.2018
%%

% Preamble
\tytul{Beskid}{sł. muz. Andrzej Wierzbicki}{SETA}
\begin{text}
    A w Beskidzie rozzłocony buk,\\
    A w Beskidzie rozzłocony buk.\\
    Będę chodził Bukowiną\\
	z dłutem w ręku by w dziewczęcych twarzach\\
    Uśmiech rzeźbić,\\
    niech nie płaczą już\\
    Niech się cieszą po kapliczkach moich dróg.

    \vin Beskidzie, malowany cerkiewny dach,\\
    \vin Beskidzie, zapach miodu w bukowych pniach.\\
    \vin Tutaj wracam, gdy ruda jesień\\
    \vin Na przełęcze swój tobół niesie,\\
    \vin Słucham bicia dzwonów w przedwieczorny czas.

    \vin Beskidzie, malowany wiatrami dom,\\
    \vin Beskidzie, tutaj słowa inaczej brzmią,\\
    \vin Kiedy krzyczą w jesienną ciszę,\\
    \vin Kiedy wiatrem szeleszczą liście,\\
    \vin Kiedy wolność się tuli w ciepło moich rąk,\\
    \vin Gdy jak źrebak się tuli do moich rąk.

    A w Beskidzie zamyślony czas,\\
    A w Beskidzie zamyślony czas.\\
    Będę chodził z nim poddaszem gór,\\
    By zerwanych marzeń struny\\
    Przywiązywać\\
    niespokojnym dłoniom drzew,\\
    Niech mi grają na rozstajach moich dróg.

    \vin Beskidzie...
\end{text}
\begin{chord}
    G CDG CD\\
	G C e a D^7\\
	C D^7\\
	G C\\
	G\\
    C C^{7+} a D^7\\
	C D G C D
	
	G C D G\\
	G C H^7 e\\
	C D^7\\
	G C\\
	G C C^{7+} a D^7\\
	(C D G)
	
\end{chord}