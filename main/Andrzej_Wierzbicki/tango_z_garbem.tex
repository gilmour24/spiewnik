\tytul{Tango z garbem}{sł. muz. Andrzej Wierzbicki}{WAR}
\begin{text}
    Dawno minęły czasy, gdy matka,\\
    Łzy kryjąc w kącie z cicha ronione,\\
    Wsuwała chyłkiem ci do plecaka,\\
    Bułeczki z serem i salcesonem.

    Tak wyruszałeś na swoje pierwsze,\\
    Te wymarzone, wyśnione szlaki,\\
    Ludziom kroczyłeś ufnie naprzeciw,\\
    Nagle słyszałeś szydercze takie:

    \vin To ten wariat z garbem,\\
    \vin od tego się nie umiera,\\
    \vin To ten wariat z garbem,\\
    \vin co turystycznie spędza czas.\\
    \vin To ten wariat z garbem,\\
    \vin który po drogach kurz wyciera,\\
    \vin To ten wariat z garbem,\\
    \vin który próbuje schwytać wiatr. |x2

    Potem mijały lata, ty ciągle,\\
    Szukałeś marzeń na szarych drogach,\\
    Struną dzwoniące, kurzem przyćmione,\\
    Buty zmieniane na twoich nogach.

    Nic nie mówiłeś, widząc na niebie,\\
    Ptaków wędrownych tajemne znaki,\\
    Potem wracałeś niby do siebie,\\
    I znów słyszałeś szydercze takie:

    \vin To ten wariat...

    Dzisiaj stateczny ojciec rodziny,\\
    Samochód, żona, dzieci, M4,\\
    Brydż u znajomych i imieniny,\\
    Spacer w niedzielę, trochę opery.

    Czasem za oknem widzisz, jak idą,\\
    Dźwigając swoje wielkie plecaki,\\
    Chciałbyś dołączyć, stanąć w szeregu,\\
    I znów usłyszeć szydercze takie:

    \vin To ten wariat...
\end{text}
\begin{chord}
    d A d D^7\\
    g D^7 g\\
    A A^7 d\\
    E E^7 A

    d A d D^7\\
    g D^7 g\\
    A A^7 d\\
    E E^7 A

    d\\
    A^7 d\\
    D^7\\
    g\\
    A\\
    A^7 d\\
    E\\
    E^7 A^7\\
    (A^7 d)
\end{chord}