\tytul{Ballada o wilku}{Piotr Bułas}{Ze Starej Szuflady}
\begin{text}
\begin{footTwo}
Tamtą zimę wciąż pamiętam, już nie ważne który rok\\
Tak jak złapał mróz na święta, trzymał aż po Wielką Noc\\
Śnieg zasypał okolicę, mróz przeszywał dom na wskroś\\
Głodnych wilków smętne wycie słychać było dzień i noc

Wyjeżdżałem gdzieś o świcie, zimny silnik ciągle gasł\\
Przeklinałem świat i życie, żeby wraka trafił szlag\\
Wtedy z zaspy gdzieś wyskoczył, prosto mi pod koła wpadł\\
Patrząc śmierci prosto w oczy, wilk, co dość już życia miał

\vin Każdy z nas wpisany ma ten dzień\\
\vin Kiedy kończy się już czas, gdy trzeba odejść w cień\\
\vin Trzeba iść przez życie twardo, nie żałować łez i sił\\
\vin Śmierci w oczy spojrzeć hardo, ze spokojem, jak ten wilk

Stary Janek nas opuścił, ciężko było kopać grób\\
Dom zostawił całkiem pusty, nie dla niego taki mróz\\
Janek życie miał nielekkie, przeżył wojny wszystkie trzy\\
Odszedł cicho i bez lęku, odszedł z dumą jak ten wilk

Tamtą zimę zapamiętam, zasypało cały las\\
Głodne wilki wyły smętnie, ktoś nie wracał raz po raz\\
Ale śni mi się po nocach, jak przed maską dumnie stał\\
Patrząc śmierci prosto w oczy, wilk, co dość już życia miał

\vin Każdy z nas wpisany ma ten dzień\\
\vin Kiedy kończy się już czas, gdy trzeba odejść w cień\\
\vin Trzeba iść przez życie twardo, nie żałować łez i sił\\
\vin Śmierci w oczy spojrzeć hardo, ze spokojem, jak ten wilk\\
\vin Z głową dumnie podniesioną, jednym krokiem przejść przez próg\\
\vin Z drugiej strony, za zasłoną, czeka na nas Dobry Bóg.
\end{footTwo}
\end{text}
\begin{chord}
\begin{footTwo}
C $\mathrm{F^{7+}}$ C\\
C $\mathrm{F^{7+}}$ G\\
F G C a\\
d G F G

C $\mathrm{F^{7+}}$ C\\
C $\mathrm{F^{7+}}$ G\\
F G C a\\
d G F G

F G C a\\
d G C $\mathrm{C^{7}}$\\
F G C a\\
d G F G

C $\mathrm{F^{7+}}$ C\\
C $\mathrm{F^{7+}}$ G\\
F G C a\\
d G F G

C $\mathrm{F^{7+}}$ C\\
C $\mathrm{F^{7+}}$ G\\
F G C a\\
d G F G

F G C a\\
d G C $\mathrm{C^{7}}$\\
F G C a\\
d G F G\\
F G C a\\
d G F G
\end{footTwo}
\end{chord}
