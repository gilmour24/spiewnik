\tytul{Beskid jesienią (Hora)}{sł. Wawrzyniec Hubka, muz. Jan Busz}{Elekt}
\begin{textn}
    Już pustką świecą hale\\
    samotny gnie się świerk\\
    Na granitowej skale\\
    zmęczony rogacz legł

    Słoneczko ledwo wstanie\\
    a już się kładzie spać\\
    Już tylko wiatr na polanie\\
    zaczyna liście gnać

    \vin O już się złoci hora\\
    \vin Gromadki szuka ptak\\
    \vin Już to jesienna pora\\
    \vin I bliskiej zimy znak

    Buki jak ogień gorzą\\
    modrzewiem pachnie wrzos\\
    Rano wschodzącą zorzę\\
    liliowy wita wrzos

    Szałasy smutnie stoją\\
    ogień nie trzaska w nich\\
    Pasterze już nie poją\\
    w źródłach owieczek swych
    
    \vin O już się złoci hora...

    Ucichła już trąbita\\
    juhasów umilkł śpiew\\
    Tylko na halnych szczytach\\
    wiatr śpiewa pośród drzew

    Z południa Tatry dumne\\
    patrzą na Beskid mój\\
    A chmurki jak rozumne\\
    zdobią go w złoty strój
    
    \vin O już się złoci hora...
\end{textn}
\begin{chordw}
    a e a\\
    C d E^7\\
    F A^7 d\\
    F E^7 a

    a e a\\
    C d E^7\\
    F A^7 d\\
    F E^7 a (e G)

    a G a e\\
    F G a e\\
    a G a e\\
    F G D\\
    (a e a d F E^7)
\end{chordw}