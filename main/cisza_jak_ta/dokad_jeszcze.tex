\tytul{Dokąd jeszcze…}
{sł. Wiesława Kwinto-Koczan, muz. Michał Łangowski}
{Cisza jak ta}
\begin{text}
	\hfill\break
	\hfill\break
	Proszę nie chwiej suchym drzewkiem mojej wiary\\
	Nie mów kiedyś, nie mów nigdy, nie mów może \\
	Winne jabłka co się na nim rumieniły\\
	Dawno spadły ale przecież bywa gorzej…\\

	Dokąd jeszcze w głębi ziemi tkwią korzenie\\
	A gałęzie wyciągają się do słońca\\
	Nie odbieraj mi nadziei na spełnienie\\
	tej nadziei co się zawsze tli do końca\\

	Dokąd jeszcze w głębi ziemi tkwią korzenie\\
	A gałęzie wyciągają się do słońca\\
	Nie odbieraj mi nadziei na spełnienie\\
	tej nadziei co się zawsze tli do końca\\

	Bardzo proszę… nie zamykaj tamtej furtki\\
	Co nas wiodła tyle razy do ogrodu\\
	Może razem ocieplimy znów jabłonie\\
	Żeby zimą nie musiały drżeć od chłodu\\

	Dokąd jeszcze w głębi ziemi tkwią korzenie\\
	A gałęzie wyciągają się do słońca\\
	Nie odbieraj mi nadziei na spełnienie\\
	tej nadziei co się zawsze tli do końca\\

	Dokąd jeszcze w głębi ziemi tkwią korzenie\\
	A gałęzie wyciągają się do słońca\\
	Nie odbieraj mi nadziei na spełnienie\\
	tej nadziei co się zawsze tli do końca\\

	Dokąd jeszcze jest nadzieja, że na wiosnę\\
	Jak co roku szystko wokół się odmieni\\
	Pozwól wierzyć, że pójdziemy do ogrodu,\\
	gdzie się nasze suche drzewko zazieleni\\
\end{text}
\begin{chord}
	G fis e A\\
	G D e A\\
	D G\\
	e G A\\
	D G\\
	e G A\\

	G D\\
	h A\\
	G D\\
	h A\\

	G D\\
	h A\\
	G D\\
	h A D\\

	D G\\
	e G A\\
	D e\\
	G A\\

	G D\\
	h A\\
	G D\\
	h A\\

	G D\\
	h A\\
	G D\\
	h A D\\
	
	G D\\
	h A\\
	G D\\
	h A D
\end{chord}
