%%
%% Author: bartek.rydz
%% 14.02.2019
%%
% Preamble
\tytul{Biała baśń}{sł. Mirosław Ostrycharz, m. Robert Leonhard}{Małżeństwo z Rozsądku}
\begin{text}
    W osypanych śnieżnym puchem\\
    Koleinach leśnych dróg\\
    Wznieca wiatr nagłym podmuchem\\
    Burze lśniących bielą smug.

    Świt przejrzysty, wklęty w ciszę,\\
    Zatopiony w śnieżną jaśń,\\
    Blaskiem słońca w brzozach pisze\\
    Oszronioną, mroźną baśń. / x2

    Tak się dziwnie w oczach mieni\\
    Oszadziały, cichy las...\\
    Pójdźmy w odmęt tej przestrzeni,\\
    W której dotąd brakło nas.

    Niech nas mrozem wiatr owionie...\\
    Nim zaginie po nas ślad,\\
    Razem z nami w snach utonie\\
    Zbłękotniały, śnieżny świat.

    Zagubimy się w dąbrowie,\\
    W białych słońcach, w siwych mgłach\\
    I nikt o nas się nie dowie,\\
    Gdzieśmy byli, w jakich snach...

    Zabłąkanych w mroźnych świtach,\\
    Wracających z białych głusz\\
    W kruchych brzozach nas przywita\\
    Lśniący słońcem srebrny kurz...
\end{text}
\begin{chord}
    h cis fis\\
    $\mathrm{D^{7+}}$ E $\mathrm{A^{7+}}$\\
    $\mathrm{D^{7+}}$ cis\\
    $\mathrm{D^{7+}}$ cis fis

    h cis fis\\
    $\mathrm{D^{7+}}$ E $\mathrm{A^{7+}}$\\
    $\mathrm{D^{7+}}$ cis\\
    $\mathrm{D^{7+}}$ $\mathrm{cis^{E}}$ $\mathrm{fis^{E}}$

    A E\\
    h fis\\
    D A\\
    h cis $\mathrm{D^{7+}}$
\end{chord}
