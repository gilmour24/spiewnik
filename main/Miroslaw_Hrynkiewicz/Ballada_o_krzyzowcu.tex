\tytul{Ballada o Krzyżowcu}{sł. muz. M. Hrynkiewicz}{Mirosław Hrynkiewicz}
\begin{text}
    Wolniej, wolniej, wstrzymaj konia,\\
    dokąd pędzisz w stal odziany?\\
    Pewnie tam, gdzie błyszczą w dali\\
    Jeruzalem białe ściany?\\
    Pewnie myślisz, że w świątyni\\
    zniewolony Pan twój czeka,\\
    żebyś przyszedł Go ocalić,\\
    żebyś przybył doń z daleka?

    \vin Na, na, na, na...

    Wolniej, wolniej, wstrzymaj konia,\\
    byłem dzisiaj w Jeruzalem.\\
    Przemierzałem puste sale,\\
    Pana twego nie widziałem.\\
    Pan opuścił Święte Miasto\\
    przed minutą, przed godziną,\\
    w chłodnym gaju za murami\\
    z Mahometem pije wino.
        
    \vin Na, na, na, na...

    Wolniej, wolniej, wstrzymaj konia,\\
    chcesz oblegać Jeruzalem?\\
    Strzegą ją wysokie wieże,\\
    strzegą ją mahometanie.\\
    Pan opuścił Święte Miasto,\\
    na nic poświęcenie twoje.\\
    Po cóż niszczyć białe mury,\\
    Po co ludzi niepokoić.
    
    \vin Na, na, na, na...

    Wolniej, wolniej, wstrzymaj konia,\\
    porzuć walkę niepotrzebną.\\
    Porzuć miecz i włócznię swoją\\
    i jedź ze mną i jedź ze mną.\\
    Bo gdy szlakiem ku północy\\
    podążają hufce ludne,\\
    ja podnoszę dumnie głowę\\
    i odjeżdżam na południe.

    \vin Na, na, na, na...    
\end{text}
\begin{chord}
    e\\
    A\\
    C\\
    D\\
    e\\
    A\\
    C\\
    D

    e A C D e
\end{chord}
