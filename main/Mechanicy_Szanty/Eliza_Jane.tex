%%
%% Author: EL Profesor
%% %26.05.2020
%%
\tytul{Eliza Jane}{sł. H.Czekała, L.Klupś, muz.trad.}{Mechanicy Shanty}
\begin{text}
   
    Do portu przybił statek nasz\\       
    W słoneczny dzień.\\                 
    Po długim rejsie sam to znasz,\\   
    Przyszedł zabawy czas.\\             
    Na bulwar, gdzie dziewczynek moc,\\ 
    Wybrałem się,\\                     
    Tam panna w oko wpadła mi,\\         
    Miałem apetyt lwi.                

    \vin O! Eliza, Eliza Jane,\\              
    \vin O! Eliza, Eliza Jane.\\              
    \vin O! Eliza, Eliza Jane,\\              
    \vin O! Eliza, Eliza Jane.\\             

    Więc klękam przed nią mówiąc tak:\\
    Ślicznotko ma,\\
    Poczujesz w ustach morza smak,\\
    Kiedy buziaka dasz.\\
    A ona na to: "Takluj się!"\\
    Wszak dobrze wiesz,\\
    Z kamieniem pierścień najpierw daj,\\
    Potem dam Ci, co chcesz.

    Oj! Dałem pierścień, forsę też,\\
    Ślicznotce mej.\\
    Przed ołtarz wkrótce wiodła mnie,\\
    Żebym nie uciekł jej.\\
    Na drugi dzień ruszałem w rejs,\\
    Bo cały świat\\
    Ma tysiąc portów, tysiąc miejsc,\\
    W każdym z nich Liza ma.
   

\end{text}
\begin{chord}
    G C\\
    G D\\
    G C\\
    G D G\\
    G C\\
    G D\\
    G C\\
    G D G

    G C G D\\
    G C G D G\\
    G C G D\\
    G C G D G

    G C\\
    G D\\
    G C\\
    G D G\\
    G C\\
    G D\\
    G C\\
    G D G

    G C\\
    G D\\
    G C\\
    G D G\\
    G C\\
    G D\\
    G C\\
    G D G
\end{chord}