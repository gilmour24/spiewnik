\tytul{Papierowy żołnierzyk}{sł. Bułat Okudżawa}{}
\begin{text}
	Raz pewien żołnierz sobie żył\\
	odważny i zawzięty,\\
	lecz cóż?... zabawką tylko był.\\
	z papieru był wycięty.

	Świat zmieniać, uszczęśliwiać go,\\
	nasz żołnierz był gotowy,\\
	lecz nad łóżeczkiem wisiał, bo\\
	był tylko papierowy.

	Pod kule chętnie by, jak w dym\\
	szedł za nas bez namowy,\\
	i mieliśmy sto pociech z nim:\\
	bo był on papierowy.

	I nigdy mu nie zwierzał sztab\\
	tajemnic swych wojskowych.\\
	A czemu tak? Dlatego tak?\\
	bo był on papierowy.

	Wyzywał los, w pogardzie miał\\
	tchórzliwych dezerterów,\\
	i 'Ognia! Ognia!' ciągle łkał,\\
	choć przecież był z papieru.

	Niejeden wódz już w ogniu znikł,\\
	niejeden szeregowy...\\
	I poszedł w ogień...\\
	Spłonął w mig - żołnierzyk papierowy.
\end{text}
\begin{chord}
	a\\
	a E\\
	E\\
	E^7 a

	a\\
	a E\\
	E\\
	E^7 a

	a\\
	a E\\
	E\\
	E^7 a

	a\\
	a E\\
	E\\
	E^7 a

	a\\
	a E\\
	E\\
	E^7 a

	a\\
	a E\\
	E\\
	E^7 a
\end{chord}
