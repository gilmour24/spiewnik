\tytul{Zawrócić}{sł. Wiesława Kwinto-Koczan, muz. G. Śmiałowski}{Bez Zobowiązań}
\begin{text}
Za cugle trzeba złapać czas,\\
do tyłu mu wykręcić głowę.\\
Zawrócić, żeby jeszcze raz\\
do szczęścia znaleźć swą podkowę.

\vin Popasów kilka wstecz,\\
\vin zakrętów losu kilka,\\
\vin poprawić każdą rzecz,\\
\vin wystarczy tylko chwilka.

Spieniony w biegu czasu pysk\\
zanurzyć w chłodną wodę,\\
iskier spod kopyt złoty błysk\\
niech wróżbą będzie na pogodę.

Rozwianą grzywę chwycić w dłonie\\
i choć się spadło już nie raz,\\
spróbować – jak spłoszone konie\\
- zatrzymać ten pędzący czas.
\end{text}
\begin{chord}
    a G $\mathrm{F^{7+}}$\\
    a G $\mathrm{F^{7+}}$\\
    a G $\mathrm{F^{7+}}$ C\\
    d E* E

    $\mathrm{F^{7+}}$ C\\
    $\mathrm{F^{7+}}$ C\\
    $\mathrm{F^{7+}}$ C\\
    d E* E\\
    (d e F G)
\end{chord}
