\tytul{Landrowerowa serenada}{sł.J.H.Gutwiński, muz.Robert Marcinkowski}{Bez Zobowiązań}
\begin{text}
    \hfill\break
    \hfill\break
Majowa panna w landrowerze\\
To bierze mnie, bo właśnie maj\\
Majowa panna w landrowerze\\
To bierze mnie, bo właśnie maj\\
Za szybą bardzo dziki teren\\
A w perspektywie dziki kraj

Ładuje napęd w cztery koła\\
Tego mi trzeba, w garach gra\\
Za kółkiem ona napalona\\
Hit majowego dnia

Majowa panna w landrowerze\\
To bierze mnie, gdy kwitną bzy\\
W butelce pustka po horyzont\\
Krajobraz szkła majowe łzy
\end{text}
\begin{chord}
$\mathrm{E^7}$ $\mathrm{A^7}$ $\mathrm{H^7}$ |x2

    $\mathrm{E^7}$\\
    A $\mathrm{H^9}$ $\mathrm{E^7}$\\
    $\mathrm{A^7}$\\
    $\mathrm{A^7}$ $\mathrm{C^9}$ $\mathrm{E^7}$\\
    $\mathrm{H^7}$\\
    A $\mathrm{C^9}$ $\mathrm{H^9}$ $\mathrm{E^7}$\\
    $\mathrm{A^7}$ H
\end{chord}
