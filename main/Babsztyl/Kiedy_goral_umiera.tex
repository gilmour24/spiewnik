%%
%% Author: bartek.rydz
%% 12.02.2019
%%
% Preamble
\tytul{Kiedy góral umiera (Góralska opowieść)}{sł. muz. P. Kasperczyk}{Babsztyl}
\begin{text}
    \begin{smallTwo}
        Kiedy góral umiera, to góry z żalu sine\\
        Pochylają nad nim głowy, jak nad swoim synem.\\
        Las w oddali szumi mu odwieczną pieśń bukową,\\
        A on długo sposobi się przed najdalszą drogą.

        Kiedy góral umiera, to nikt nad nim nie płacze,\\
        Siedzi, czeka aż kostucha w okno zakołacze.\\
        Oczy jeszcze raz podniesie wysoko do nieba,\\
        By pożegnać góry swoje, by im coś zaśpiewać.

        \vin Góry moje, wierchy moje, otwórzcie swe ramiona,\\
        \vin Niech na miękkim z mchu posłaniu\\
        \vin Cichuteńko skonam.\\
        \vin Ojcze mój, Halny Wietrze, powiej ku północy,\\
        \vin Ciepłą, drżącą swoją ręką zamknij zgasłe oczy,\\
        \vin Bym mógł w ziemię wrosnąć, strzelić potem\\
        \vin Do słońca smreczyną\\
        \vin I na zawsze szumieć już nad moją dziedziną.

        Kiedy góral umiera, to dzwony mu nie grają,\\
        Cicho wspina się pod bramy góralskiego raju,\\
        Tylko strumień na kamieniach żałobną nutę składa,\\
        Tylko nocka chmurnooka górom opowiada.

        A gdy góral już umrze, to nikt nie układa baśni,\\
        Tylko w niebie roziskrzonym mała gwiazdka gaśnie.\\
        Ziemia twardą, szorstką ręką tuli go do siebie,\\
        By na zawsze już mógł zostać pod góralskim niebem.
\end{smallTwo}
\end{text}
\begin{chord}
    \begin{smallTwo}
    D D^{7}\\
    G D\\
    e G D\\
    e G D

    D D^{7}\\
    G D\\
    e G D\\
    e G D

    D e\\
    G\\
    D\\
    D e\\
    G D\\
    e\\
    G D\\
    e G D
\end{smallTwo}
\end{chord}