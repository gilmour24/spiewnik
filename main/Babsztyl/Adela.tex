%%
%% Author: EL Profesor
%% %26.09.2020
%%
\tytul{Adela}{słowa: Bogusław Choiński, Jan Gałkowski muzyka: Jerzy "Duduś" Matuszkiewicz}{Babsztyl}
\begin{text}
    Adela już zakłada suknię cienką\\
    Na wiosnę kwiatki rosną i kwitnie miesiąc maj\\
    Gdy spytasz ją dla kogo ta sukienka?\\
    Dla kochasia, który odszedł w siną dal.

    \vin W siną dal, w siną dal\\
    \vin To dla kochasia, który odszedł w siną dal\\
    \vin W siną dal, w siną dal\\ 
    \vin To dla kochasia, który odszedł w siną dal.

    Czerwone ma podwiązki pod tą suknią\\
    Na wiosnę kwiatki rosną i kwitnie miesiąc maj\\
    Gdy spytasz ją dla kogo taki luksus?\\
    To dla kochasia, który odszedł w siną dal.

    U drzwi, u drzwi, jej tatuś czyści spluwę\\
    Na wiosnę kwiatki rosną i kwitnie miesiąc maj\\
    Gdy spytasz ją na czyją to jest zgubę?\\
    To na kochasia, który odszedł w siną dal.  

    Nad łóżkiem ślub stwierdzony rejentalnie\\
    Na wiosnę kwiatki rosną i kwitnie miesiąc maj\\
    Gdy spytasz ją z kim żyjesz tak moralnie?\\
    To z kochasiem, który odszedł w siną dal.

    \vin W siną dal, w siną dal\\
    \vin Pomaszerował lewa prawa w siną dal\\
    \vin W siną dal, w siną dal\\ 
    \vin Pomaszerował lewa prawa w siną dal.

\end{text}
\begin{chord}

    G C G\\
    G e A^7 D\\
    G G^7 C G\\
    G D^7 G

    G C \\
    G D^7 G \\
    G C \\
    G D^7 G

    G C G\\
    G e A^7 D\\
    G G^7 C G\\
    G D^7 G

    G C G\\
    G e A^7 D\\
    G G^7 C G\\
    G D^7 G

    G C G\\
    G e A^7 D\\
    G G^7 C G\\
    G D^7 G

    G C \\
    G D^7 G \\
    G C \\
    G D^7 G

\end{chord}