%%
%% Author: EL Profesor
%% %26.05.2020
%%
\tytul{Irlandzki Wędrowiec}{sł H.Czekała, muz. trad.}{Mechanicy Shanty}
\begin{text}
   
    Raz! Dwa! Trzy!\\
    I do przodu się rwał przez fale nasz ship,\\
    Szesnasty to chyba był rok.\\
    Zły Przylądek Horn po nocach się śnił,\\
    Za rufą gdzieś został New York.\\
    Był piękny, jak panna przed wyjściem na bal,\\
    Pulchniutki, krąglutki miał zadek.\\
    Dwadzieścia trzy maszty sterczały do chmur,\\
    Kto zgadnie, jak zwał się ten statek?

    Był tam Barney McGee, gdzieś z wybrzeża Lee\\
    I Hogan, co w County miał swój bar.\\
    Był tam John McGurk, no i Paddy Malone,\\
    Co batem do pracy nas gnał.\\
    Bill Casey, ten pijak i szuler jak nikt,\\
    Niestety na pokład wlazł w Dover.\\
    O reszcie nie wspomnę, bo zbraknie mi sił,\\
    Brał wszystkich "Irlandzki Wędrowiec".

    Ładunek zalegał od topu do dna,\\
    Sam nie wiem, jak mógł zmieść się.\\
    Prócz koni i kur, stu beczek bez dna,\\
    Upchnięto po kątach co złe.\\
    Te sześć milionów bel bawełny, co na dnie\\
    Leżała w wodzie przez całe lata\\
    I trzy miliony świń, i sześć milionów psów -\\
    Sam diabeł nam figla tu spłatał.

    Lecz stracił statek nasz swą drogę we mgle,\\
    Przez sztorm, który przyniósł mu śmierć.\\
    Z załogi tylko dwóch wytrwało po kres,\\
    Nim złożył się lekko na dnie.\\
    Zdradliwej skały ząb zakończył długi rejs,\\
    Kapitan, podły tchórz, skoczył w morze,\\
    Zostałem tylko ja, by wszystko to wam rzec...\\
    (Raz! Dwa! Trzy!)\\
    I jak zginął "Irlandzki Wędrowiec".
  

\end{text}
\begin{chord}
    \chordfill
    G e C\\
    G D \\
    G e C\\
    G D G\\
    G D\\
    G D \\
    G e C\\
    G D G

    G e C\\
    G D \\
    G e C\\
    G D G\\
    G D\\
    G D \\
    G e C\\
    G D G

    G e C\\
    G D \\
    G e C\\
    G D G\\
    G D\\
    G D \\
    G e C\\
    G D G

    G e C\\
    G D \\
    G e C\\
    G D G\\
    G D\\
    G D \\
    G e C\\
	\chordfill
    G D G
\end{chord}